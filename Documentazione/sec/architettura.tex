\section{Architettura del sistema}

\subsection{Struttura del progetto}
Il codice del progetto è disponibile all'interno della cartella \texttt{triphippie}, al suo interno vi sono tre sotto-cartelle principali:
\begin{itemize}
    \item \textbf{src}: contenente il codice Java;
    \item \textbf{target}: contenente i file di compilazione e l'eseguibile;
    \item \textbf{resources}: contenente le immagini relative al profilo di ogni utente.
\end{itemize}
La cartella \texttt{resources} contiene al suo interno una sotto-cartella per ogni utente registrato, al suo interno saranno inserite tutte le risorse relative a tale utente.\\
La cartella \texttt{src} contiene sia il codice dell'applicazione che quello relativo ai test.

\begin{figure}[h]
    \centering
    \begin{minipage}{7cm}
        \dirtree{%
        .1 triphippie.
        .2 resources.
        .3 user\_id.
        .3 \dots.
        .2 src.
        .3 main.
        .3 test.
        .2 target.
        }
    \end{minipage}
\end{figure}

\noindent
Nella cartella \texttt{main} sono presenti le sotto-cartelle:
\begin{itemize}
    \item \textbf{java}: contenente i pacchetti in cui è stata suddivisa l'applicazione;
    \item \textbf{resources}: contiene i file di configurazione come \texttt{.env}, \texttt{application.properties} ed il file \texttt{compose-dev.yaml} per la creazione del database di sviluppo.
\end{itemize} 
\begin{figure}[h]
    \centering
    \begin{minipage}{7cm}
        \dirtree{%
        .1 main.
        .2 java.
        .3 com.synclab.triphippie.
        .2 resources.
        .3 \textit{.env}.
        .3 \textit{application.properties}.
        .3 \textit{compose-dev.yaml}.
        }
    \end{minipage}
\end{figure}
\noindent
Nella cartella \texttt{java} sono presenti i package in cui è stato suddiviso il progetto.
Il pacchetto principale è \texttt{com.synclab.triphippie}, al suo interno sono presenti ulteriori pacchetti che racchiudono le diverse tipologie di classi create.

\begin{figure}[h]
    \centering
    \begin{minipage}{7cm}
        \dirtree{%
        .1 com.synclab.triphippie.
        .2 config.
        .2 controller.
        .2 dto.
        .2 exception.
        .2 model.
        .2 repository.
        .2 service.
        .2 util.
        } 
    \end{minipage}
\end{figure}

\noindent
I pacchetti suddividono il progetto individuandone le componenti proncipali, ossia controller, servizi, DTO (Data Transfer Object), repository.\\
Il modello viene usato per definire la struttura del database SQL, dunque non si ha un file \texttt{schema.sql}.\\
\noindent
Ulteriori pacchetti sono:
\begin{itemize}
    \item \texttt{exception}: definisce eccezioni personalizzate ed un \texttt{GlobalExceptionHandler};
    \item \texttt{util}: definisce i mapper per i DTO, le classi per la gestione dell'hashing delle password e per la generazione dei token JWT;
    \item \texttt{config}: contiene classi di configurazione, in particolare quella dedicata alla gestione del Cors Mapping.
\end{itemize}
