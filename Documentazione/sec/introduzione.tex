\newpage
\section{Introduzione}

\subsection{Contatti}
\textbf{Studente:} \Author, \StudentMail \\
\textbf{Tutor aziendale:} \TutorAziendale, \EmailTutorAziendale \\

\subsection{Scopo del documento}
La seguente documentazione fornisce istruzioni per l'utilizzo software prodotto, durante il tirocinio curricolare, presso l'azienda Sync Lab.

\subsection{Informazioni sul progetto}
Lo scopo del progetto è la realizzazione del back-end relativo all'applicazione denominata Trip Hippie, il progetto è stato svolto simultaneamente ad altri studenti che si sono occupati della realizzazione del front-end e di un servizio relativo alla chat.\\
Il prodotto da me realizzato è disponibile nella seguente repository: \href{https://github.com/AlessandroTigani/ProgettoStageSyncLab}{Progetto Sync Lab}.
Al suo intenro sono presenti i file relativi alla documentazione Stopligh, la cartella contenente il codice del programma, il file necessario per generare il database ed avviare l'applicazione tramite Docker.\\
Il progetto si compone di due parti principali, una incentrata sugli utenti ed una sulla gestione viaggi.
Non si è scelto di utilizzare una architettura a microservizi per questioni relative a necessità di semplicità applicativa e di tempistiche, si dispone quindi di un unico progetto che include tutte le funzionalità.

\subsection{Funzionalità}
Il back-end da me realizzato permette di effettuare operazioni inerenti la gestione di utenti e viaggi.
Le funzionalità inerenti le chat sono state sviluppate da uno studente differente e non sono state implementate nel presente progetto.\\
Le funzionalità sono state raggruppate in due gruppi principali, User e Trip.
La documentazione relativa a tutte le operazioni da rendere disponibili mi è stata consegnata utilizzando StopLight.
Pur non avendo creato un progetto pubblico è possibile importare la documentazione tramite due file presenti nel root del \href{https://github.com/AlessandroTigani/ProgettoStageSyncLab}{Repository}, denominati:
\begin{itemize}
    \item Trip.json
    \item User.json
\end{itemize}
Possono esserci differenze tra la versione presente nella mia repository e quella in possesso di altri studenti, in quanto durante il mio tirocinio mi è stato richiesto di aggiungere anche un servizio necessario alla registrazione delle preferenze dei singoli utenti.